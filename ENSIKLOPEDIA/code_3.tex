\chapter{Radiasi Batu}

[Cesssssssss.... pluk.. blubuk blubuk blubuk... pufff... zzzz ctang.. shuw shuw shuw.. clang clang tak clang shuw hffubuff] 
remang remang di sebuah ruangan seperti laboratorium kimia. 
Dia mengenakan jass putih terlihat sedang melakukan eksperimen. 
dengan botol petri dan gelas gelas kaca yg berisi cairan berbagai warna, ada yg bercahaya, ada yang berasap, ada yang bergelembung seakan mendidih. 
di samping gelas gelas kaca tersebut terdapat ember kayu berisi batu batu kristal dengan berbagai ukuran dan warna pula, ada juga yang bersinar, ada juga yang terlihat seakan hanyalah kumpulan batu krikil. 
Dia memasukan batu kristal seukuran jari kelingngking berwarna biru ke gelas berisi cairan bening. 
[sfx] 
kemudian gelas kaca tersebut pecah dan batu kristal tersebut terpental pental kesegala arah penjuru ruangan itu sembari merusak apa yag tersenggol. 
Ketika batu itu melayang ke arah wajah orang yang bertanggung jawab atas kejadian tersebut dia menghadangnya dengan sebuah bantal sebagai perisai dari lontaran batu kerikil liar itu.. 
[buff..]
batu tersebut akhirnya berhenti dari pantulan liarnya dan melayang seakan tidak ada gravitasi di ruangan tersebut. 
Orang tersebut kemudian menyentuh batu itu dengan penasaran untuk kemudian ia tersenyum,.. 
"Berhasil.. >:>"

iya kemudian melanjutak ekperimennya dengan batu dengan warna lain, batu merah terbakar. kuning memancarkan listrik. putih mengkompresi udara sekitar hingga mebentuk kiristal baru yang segera meledak beberapa saat setelah terbentuk. kemudian kristal biru yang bereaksi dengan gravitasi H20 disekitar melayang.

1 minggu kemudian,..

air craft/armor?