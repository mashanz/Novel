\chapter{TERDAMPAR}

\textbf{plot: (karakter)} Unknowen Ligh jump in star gate(black hole)

\textbf{plot: (tempat)} angkasa, bulan, suhu minus celcius

\textbf{plot: (teknologi di sini)} pesawat induk, rusak berat, chrash di bulan

\textbf{plot: (bioteknologi)} none

\textbf{plot: (relasi ras)} none

ceritnya\\
======================================\\
aksi 1: se saat sebelum jumping\\
======================================\\

Hening gelap malam dan cerah bintang seakan menghiasi sunyinya keindahan angkasa luar. Percikan peluru cahaya menebas langit terlihat seperti bintang jatuh. Ledakan mesin bagaikan kembang api angkasa di kejauhan. Sunyi kelap dari kejauhan karena tidak ada udara yang menghantarkan suara, namun bising mesin dan alarem menemani sunyi didalamm kapal yang melakukan parade kembang api antariksa. Para kru kapal berteriak ke sana kemari, menyaut teriakan kapten kapal.

Sebuah kapal usang di ikuti beberapa kapal dengan cahaya lampu merah biru menyebrangi samudra angkasa. Saling kejar seakan tidak ada yang mau meninggalkan dan ditinggalkan. Kapal usang terlihat semakin mirip rongsokan kaleng yang membara, dilemparkan di tengah kegelapan malam.

Badan kapal sebelah kiri sudah hancur. Tekanan udara udara lambung tengah semakin menurun. Mesin pendorong sebelah kanan tidak merespon. Tangki bahan bakar nuklir tidak kuat lagi menahan panas reaksi. Komputer kendali menjadi berisik dengan berbagai peringatan error. Seakan tidak ada harapan, protokol percobaan yang baru di pasang sebelum lepas landas diaktifkan. tidak ada yang tahu kalo prosedur ini berfungsi atau hanya sampah.

Seketika alarem peringatan semakin ramai. Mengumumkan protokol perintah yang sedang di jalankan merupakan protokol asing dan berbahaya.

[PERINGATAN... protokol asing akan di eksekusi... tidak ada daya akselerator yang cukup pada sumbu rudal... aksi dapat menyebabkan ledakan grafitasi... lubang hitam di konfirmasi... tarikan grafitasi berlebih...]

Dengan menyentuh layar bertulis abaikan peringatan, kapal menjadi sunyi.

[Eksekusi gerbang antar bintang... peluncuran rudal grafitasi akan di lakukan dalam hitungan 3.. 2... 1... rudal di luncurkan... KESALAHAN... rudal akan meledak pada jarak berbahaya...]

[mengaktifkan pelindung grafitasi... PERINGATAN... Daya mesin pendorong berlebih... melakukan protokol pelepasan daya otomatis...]


.\\
======================================\\
aksi 2: saat jumping\\
======================================\\
.\\
======================================\\
aksi 3: setelah jumping\\
======================================\\
.\\
======================================\\
aksi 4: terombang ambing di angkasa\\
======================================\\
.\\
======================================\\
aksi 5: intervensi grafitasi\\
======================================\\
.\\
======================================\\
aksi 6: terjatuh di bulan\\
======================================\\
.\\
======================================\\
aksi 7: pesawat hancur\\
======================================\\
.\\
======================================\\
aksi 8: selamat dengan protokol keamanan\\
======================================\\
.\\
======================================\\
aksi 9: luka parah\\
======================================\\
.\\
======================================\\
aksi 10: restorasi penyembuhan\\
======================================\\