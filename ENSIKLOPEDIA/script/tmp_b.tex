\chapter{BASIS DATA}

\textbf{plot: (karakter)} biosuit(tidak full), mini drone equip

\textbf{plot: (tempat)} didalam tanah, didarat, hutan hujan, lebat tumbuhan hijau, berbatu, terjal, basah, mendung, berbukit, sungai besar, danau, pegunungan(bukan api), rad 100m pohon ditebang, dataran di ratakan.

\textbf{plot: (teknologi di sini)} pesawat kecil eksplorasi berubah jadi base, membangun base. hasil tebangan jadi A.P.M.(atomic programable mater)

\textbf{plot: (bioteknologi)} unknown, scanning fase informasi geografis rad 1 km (rough 50 km)

\textbf{plot: (relasi ras)} none

ceritnya\\
======================================\\
aksi 1: Kendala proses data.\\
======================================\\
.\\
======================================\\
aksi 2: Membuat base\\
======================================\\
.\\
======================================\\
aksi 3: mengambil data sekitar pendaratan\\
======================================\\
.\\
======================================\\
aksi 4: meratakan tanah\\
======================================\\
.\\
======================================\\
aksi 5: menggali, membangun pondasi\\
======================================\\
.\\
======================================\\
aksi 6: membangun jaringan bawah tanah\\
======================================\\
.\\
======================================\\
aksi 7: membangun pagar\\
======================================\\
.\\
======================================\\
aksi 8: menebar kamera pengamat\\
======================================\\
.\\
======================================\\
aksi 9: mengeraskan tanah\\
======================================\\
.\\
======================================\\
aksi 10: membuat atap\\
======================================\\